
\documentclass[9pt]{beamer}

\usepackage[latin1]{inputenc}
\usepackage{colortbl}
\usepackage[english]{babel}

\newcommand{\colhref}[3][blue]{\href{#2}{\color{#1}{#3}}}%


\newcommand{\myblue} [1] {{\color{blue}#1}}
\newcommand{\newauthor}[4]{
  \parbox{0.26\textwidth}{
    \texorpdfstring
      {
        \centering
        #1 \\
        \colhref{#2}{\texttt{#3}} \\
        #4 \\
      }
      {#1}
  }
}


% for code colouring
\usepackage{minted}


% beamer template
\beamertemplatetransparentcovereddynamic
\usetheme{default}

\hypersetup{%
  pdftitle={my talk: a nice talk},%
   pdfauthor={Sebastien Binet},%
  pdfauthor={Evil Me},%
  pdfauthor={Evil \& You},%
%
}

\title[my talk: a nice talk]{my talk: a nice talk}
\author[Sebastien Binet \& Evil Me \& Evil \& You]{
 \parbox{0.26\textwidth}{
	\texorpdfstring
	  {
		\centering
 		Sebastien Binet \\
 		CNRS/IN2P3 \\
 		\colhref{http://twitter.com/0xb1ns}{\texttt{@0xb1ns}} \\
 		\colhref{https://github.com/sbinet}{\texttt{https://github.com/sbinet}} \\
 	  }
	{Sebastien Binet}
}
 \and %
 \parbox{0.26\textwidth}{
	\texorpdfstring
	  {
		\centering
 		Evil Me \\
 		\colhref{mailto:evil@example.com}{\texttt{evil@example.com}} \\
 	  }
	{Evil Me}
}
 \and %
 \parbox{0.26\textwidth}{
	\texorpdfstring
	  {
		\centering
 		Evil \& You \\
 		Evil Corp. \\
 	  }
	{Evil \& You}
}
 }



\begin{document}

\frame{\titlepage
}

\part<presentation>{Main Talk}

\section[slides]{slides}

\begin{frame}[fragile]
\frametitle{A chapter}


\end{frame}

\begin{frame}[fragile]
\frametitle{A title}


\colhref{https://github.com/sbinet/present-tex}{\texttt{present-tex}} converts a \texttt{.slide} presentation to a \texttt{LaTeX/Beamer} presentation.


Here are some bullets:


\begin{itemize}
\item correctly rendered
\item but not numbered
\end{itemize}


\end{frame}

\begin{frame}[fragile]
\frametitle{\texttt{present-tex} and \texttt{code}}


Consider this simple package \texttt{github.com/me/hello}:


\begin{minted}[]{go}
package main

import (
	"fmt"
)

func main() {
	fmt.Printf("hello world\n")
}

\end{minted}


\end{frame}

\begin{frame}[fragile]
\frametitle{\texttt{present-tex} and \texttt{play}}


\texttt{present-tex} has some limited supported for \texttt{.play}:


\begin{minted}[]{go}
package main

import (
	"fmt"
)

func main() {
	fmt.Printf("hello world\n")
}

\end{minted}


\end{frame}

\begin{frame}[fragile]
\frametitle{\texttt{.code} support}


\texttt{present-tex} infers the language of a \texttt{.code} snippet based on the file extension.


Here is some \texttt{C}:


\begin{minted}[]{c}
#include <stdio.h>

int main(int argc, char **argv) {
	printf("hello world\n");
	return 0;
}

\end{minted}

And here is some \texttt{python}:


\begin{minted}[]{py}
#!/usr/bin/env python2
from __future__ import print_function
print("hello world")

\end{minted}


\end{frame}

\begin{frame}[fragile]
\frametitle{\texttt{present-tex} and images}


Images are supported, such as this lovely \texttt{PNG} gopher:


\begin{figure}[h]
\begin{center}
\includegraphics[width=3cm,height=4cm]{_figs/gopher.png}
\end{center}

\end{figure}


\end{frame}

\begin{frame}[fragile]
\frametitle{\texttt{present-tex} and images (cont'd)}


or that lovely gopher:


\begin{figure}[h]
\begin{center}
\includegraphics[width=3cm,height=4cm]{_figs/gopher.png}
\end{center}

\end{figure}


\end{frame}

\begin{frame}[fragile]
\frametitle{\texttt{present-tex} and text formatting}


\texttt{present-tex} should be able to correctly handle URLs like \colhref{https://github.com/sbinet/present-tex}{\texttt{so}}.


But, also, \textbf{bold} text and text in \emph{italics}.


Special \texttt{LaTeX} characters, such as \&\{\}\textbackslash\$\%\^{}\_\#, are also correctly handled.


\colhref{https://github.com/sbinet/present-tex}{\texttt{github.com/sbinet/present-tex}}

Snippets of code look like so:



\begin{verbatim}
$> ls /my/dir
$> exit

\end{verbatim}


\colhref{https://github.com/sbinet/present-tex}{\texttt{github.com/sbinet/present-tex}} is still a \emph{work in progress}.


\textbf{The END.}



\end{frame}

\end{document}
